%----------------------------------------------------------------------------------------
%	PAGE Conclusion
%----------------------------------------------------------------------------------------
\begin{frame}
    \frametitle{Summary of LDPE}
    \begin{block}{Advantages}
            % \item[$\blacksquare$] Low dimensional projection(LDP) approach has significance:
        \begin{itemize}
        \item Construct confidence intervals for regression coefficients
        \item Without requirement of the uniform signal strength condition
        \item Handle the situation with high-corelated design matrix. 
        % \item Avaliable for variable selection.
        % \item 
        \end{itemize}
    \end{block}

    % \begin{block}{Drawbacks}
    %     \begin{itemize}
    %         \item LDPE obtain a large confidence interval when $\eta_j, \tau_j$ is not small enough, which is given by lasso. 
    %         \item Complex computation for varibale selection. 
    %         % \item 
    %     \end{itemize}
    % \end{block}

    \vspace{24pt}

\begin{flushright}
    Thank you !
\end{flushright}

\end{frame}

\begin{frame}{Discussion}
    \begin{itemize}
        \item Why lasso to obtain $\bz_j$? What about other methods?
        \item Conditions. 
        \item Limits of Thresholded-LDPE in variable selection.
    \end{itemize}


\end{frame}
